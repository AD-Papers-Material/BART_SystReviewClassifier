% Options for packages loaded elsewhere
\PassOptionsToPackage{unicode}{hyperref}
\PassOptionsToPackage{hyphens}{url}
%
\documentclass[
]{article}
\title{Databases' search guide}
\author{}
\date{\vspace{-2.5em}2022-02-19}

\usepackage{amsmath,amssymb}
\usepackage{lmodern}
\usepackage{iftex}
\ifPDFTeX
  \usepackage[T1]{fontenc}
  \usepackage[utf8]{inputenc}
  \usepackage{textcomp} % provide euro and other symbols
\else % if luatex or xetex
  \usepackage{unicode-math}
  \defaultfontfeatures{Scale=MatchLowercase}
  \defaultfontfeatures[\rmfamily]{Ligatures=TeX,Scale=1}
\fi
% Use upquote if available, for straight quotes in verbatim environments
\IfFileExists{upquote.sty}{\usepackage{upquote}}{}
\IfFileExists{microtype.sty}{% use microtype if available
  \usepackage[]{microtype}
  \UseMicrotypeSet[protrusion]{basicmath} % disable protrusion for tt fonts
}{}
\makeatletter
\@ifundefined{KOMAClassName}{% if non-KOMA class
  \IfFileExists{parskip.sty}{%
    \usepackage{parskip}
  }{% else
    \setlength{\parindent}{0pt}
    \setlength{\parskip}{6pt plus 2pt minus 1pt}}
}{% if KOMA class
  \KOMAoptions{parskip=half}}
\makeatother
\usepackage{xcolor}
\IfFileExists{xurl.sty}{\usepackage{xurl}}{} % add URL line breaks if available
\IfFileExists{bookmark.sty}{\usepackage{bookmark}}{\usepackage{hyperref}}
\hypersetup{
  pdftitle={Databases' search guide},
  hidelinks,
  pdfcreator={LaTeX via pandoc}}
\urlstyle{same} % disable monospaced font for URLs
\usepackage[margin=1in]{geometry}
\usepackage{graphicx}
\makeatletter
\def\maxwidth{\ifdim\Gin@nat@width>\linewidth\linewidth\else\Gin@nat@width\fi}
\def\maxheight{\ifdim\Gin@nat@height>\textheight\textheight\else\Gin@nat@height\fi}
\makeatother
% Scale images if necessary, so that they will not overflow the page
% margins by default, and it is still possible to overwrite the defaults
% using explicit options in \includegraphics[width, height, ...]{}
\setkeys{Gin}{width=\maxwidth,height=\maxheight,keepaspectratio}
% Set default figure placement to htbp
\makeatletter
\def\fps@figure{htbp}
\makeatother
\setlength{\emergencystretch}{3em} % prevent overfull lines
\providecommand{\tightlist}{%
  \setlength{\itemsep}{0pt}\setlength{\parskip}{0pt}}
\setcounter{secnumdepth}{-\maxdimen} % remove section numbering
\ifLuaTeX
  \usepackage{selnolig}  % disable illegal ligatures
\fi

\begin{document}
\maketitle

\hypertarget{wos}{%
\subsection{WOS}\label{wos}}

\begin{itemize}
\tightlist
\item
  activate your institution proxy or log in in some way (it is only
  needed to get the API key, then you can log off the proxy)
\item
  go to \url{https://apps.webofknowledge.com/}
\item
  extract the API key from the URL, is denominated by SID
\item
  put the key in the secrets.R file and source it or source Setup.R
\item
  perform the search using the API
\end{itemize}

\hypertarget{ieee}{%
\subsection{IEEE}\label{ieee}}

\begin{itemize}
\tightlist
\item
  get your API key and put it into the secrets.R
\item
  source the file source Setup.R
\item
  perform the search using the API
\end{itemize}

\hypertarget{scopus}{%
\subsection{Scopus}\label{scopus}}

\begin{itemize}
\tightlist
\item
  change the NOT in the query to AND NOT
\item
  go to
  \url{https://www.scopus.com/search/form.uri?display=basic\#basic}
\item
  log in with your institution credential
\item
  set up the year range
\item
  search within: title, abstract, keywords, authors
\item
  select years in order to have just less than 2000 records
\item
  download csv (Excel). Choose ``Citation information'' and ``Abstract
  \& keywords''
\item
  call the files Scopus\#.csv with \# being a sequential number
\item
  perform\_search\_session() will find and parse the files
\end{itemize}

\hypertarget{pubmed}{%
\subsection{Pubmed}\label{pubmed}}

\begin{itemize}
\tightlist
\item
  put your API key in the secrets.R file and source it or source Setup.R
\item
  perform the search using the API
\item
  put the same query into pubmed and download records in the Pubmed
  format.
\item
  only 10k records can be downloaded at once, so use the year filter to
  partition the records
\item
  call the file Pubmed\#.nbib with \# being a sequential number
\item
  perform\_search\_session() will find and parse the files
\end{itemize}

\hypertarget{embase}{%
\subsection{Embase}\label{embase}}

\begin{itemize}
\tightlist
\item
  Login from within the institution or activate remote access
\item
  go to \url{https://www.embase.com/\#advancedSearch/default}
\item
  set your date range
\item
  in the mapping field select all apart from ``Limit to terms indexed in
  article as `major focus'\,''
\item
  in ``Sources'' select Embase
\item
  select records in blocks of 10K
\item
  chose format: Csv - fields by column, output: Full record (may take a
  while to prepare the download and you can download only one batch at
  time)
\item
  call the files Embase\#.csv with \# being a sequential number
\end{itemize}

\end{document}
